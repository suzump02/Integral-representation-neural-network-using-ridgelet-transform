\documentclass[a4paper,12pt]{article} 
\usepackage[T1]{fontenc}
\usepackage{fontspec}
\usepackage[english]{babel}
\usepackage{luatexja}
\usepackage{csquotes}
\usepackage{amsmath}
\usepackage{amssymb}
\usepackage{amsthm}
\usepackage{mathrsfs}
\usepackage{amsfonts}
\usepackage{mathtools}
\usepackage{mathabx}
\usepackage{bm}
\usepackage{color}
\usepackage{xcolor} 
\usepackage{cases}
\usepackage{cancel}
\usepackage{graphicx}
\usepackage{wrapfig}
\usepackage{threeparttable}
\usepackage{booktabs}
\usepackage{multirow}
\usepackage{enumitem}
\usepackage{array}
\usepackage{tikz}
\usepackage[margin=20mm]{geometry}
\usepackage{url}
\usepackage{ascmac}
\usepackage{cite}
%\usepackage{todonotes}

\setlength{\evensidemargin}{-2mm}
\setlength{\oddsidemargin}{-2mm}
\setlength{\topmargin}{-5mm}
\setlength{\textheight}{220mm}
\setlength{\textwidth}{165mm}
\setlength{\arraycolsep}{10pt}   
\renewcommand{\arraystretch}{1.5} 
\setlength{\jot}{2ex}

%定理環境
\newtheorem{thm}{Theorem}[section]
\newtheorem{lem}[thm]{Lemma}
\newtheorem{cor}[thm]{Corollary}
\newtheorem{prop}[thm]{Proposition}
\newtheorem{form}[thm]{Formula}
\newtheorem{df}[thm]{Definition}
\newtheorem{con}[thm]{Conjecture}

%雑多
\newcommand{\red}[1]{\textcolor{red}{#1}}
\renewcommand{\qedsymbol}{\hfill \blacksquare}
\newcommand{\Res}{\text{Res}}

%三角関数
\newcommand{\sech}{\operatorname{sech}}


%積分関係
\newcommand{\intinf}{\int_{-\infty}^{\infty}}
\newcommand{\intrm}{\int_{\mathbb{R}^m}}

%空間
\newcommand{\R}{\mathbb{R}}
\newcommand{\Y}{\mathbb{Y}^{m+1}}
\newcommand{\Sq}{\mathbb{S}}

%変換
\newcommand{\Ra}{\mathrm{R}}
\newcommand{\Rid}{\mathscr{R}}
\newcommand{\Ridd}{\mathscr{R}^{\dagger}}
\newcommand{\F}{\mathcal{F}}
\newcommand{\Lam}{\Lambda}
\newcommand{\Hil}{\mathcal{H}}

%ギリシャ記号
\newcommand{\z}{\zeta}

%ベクトル
\newcommand{\vu}{\bm{u}}
\newcommand{\va}{\bm{a}}
\newcommand{\vx}{\bm{x}}
\newcommand{\vy}{\bm{y}}
\newcommand{\vw}{\bm{w}}

%ノルム・絶対値
\newcommand{\nora}{\|\bm{a}\|}
\newcommand{\absz}{|\z|}

%微小量
\newcommand{\da}{\mathrm{d}\va}
\newcommand{\dx}{\mathrm{d}\vx}
\newcommand{\dw}{\mathrm{d}\vw}
\newcommand{\db}{\mathrm{d}b}
\newcommand{\dd}{\mathrm{d}}
\newcommand{\pd}{\partial}

%微分関係
\newcommand{\diff}[2]{\dfrac{\dd #1}{\dd #2}}
\newcommand{\diffl}[2]{\dd #1/\dd #2}
\newcommand{\pdiff}[2]{\dfrac{\pd #1}{\pd #2}}
\newcommand{\pdiffl}[2]{\pd #1/\pd #2}


%フーリエ関数
\newcommand{\psihat}{\widehat{\psi}}
\newcommand{\etahat}{\widehat{\eta}}

%特殊な関数
\newcommand{\sgn}{\mathop{\mathrm{sgn}}\nolimits}

\newcommand{\SSC}[1]{\section{#1}\setcounter{equation}{0}}


\begin{document}

\title{
\bf 
Note: On the Reconstruction Formula for the Ridgelet Transform
}
\author{
Raian Suzuki
\footnote{
Department of Physics,
Faculty of Science and Engineering,
Chuo University, 
Kasuga, Bunkyo-ku, Tokyo 112-8551, Japan;
} 
}

\date{\today (version 1.1)}

\pagestyle{plain}
\maketitle

% \begin{abstract}
  
% \end{abstract}

\clearpage

\section{Introduction}

We define the ridgelet transform and its dual as follows:

\begin{align}
  \Rid_\psi[f](\va, b) &\coloneqq \intrm f(x) \, \psi(\va \cdot \vx - b) \, \nora^s \, \dx
\end{align}

\begin{align}
  \Ridd_{\eta}[T](x) &\coloneqq \intinf \intrm  T(\va, b) \, \eta(\va \cdot \vx - b) \, \nora^{-s}\, \da \db
\end{align}

Here, we consider the reconstruction formula.

\begin{align}
  \Ridd_{\eta}[\Rid_\psi[f]](\vx) = K_{\psi, \eta} \, f(x)
\end{align}

where 

\begin{align}
  K_{\psi, \eta} &\coloneqq \intinf \frac{\overline{\psihat(\z)}\etahat(\z)}{\absz^{m}} \, \dd \z
\end{align}

\clearpage

\section{Reconstruction Formula}

Let $\eta$ and $\psi$ be fourier-transformable functions on $\R$, and let $\psihat$ and $\etahat$ be their respective Fourier transforms.
In what follows, to avoid notational conflicts, we denote the evaluation point of the reconstruction as $\vy$.

Throughout this section, 
we assume that all functions involved are sufficiently well-behaved (e.g., rapidly decaying and smooth, or belonging to appropriate $L^1$ or $L^2$ spaces) such that all interchanges of integration order are justified by Fubini's Theorem.

\begin{align}
  \Rid_\psi[f](\va, b) &= \intrm \dx \, f(x) \, \psi(\va \cdot \vx - b) \, \nora^s 
\end{align}

\begin{align}
  \Ridd_{\eta}[\Rid_\psi[f]](\vy) &= \intrm \da \intinf \db \, \left(\intrm \dx \, f(\vx) \, \psi(\va \cdot \vx - b) \, \nora^s \right)  \, \eta(\va \cdot \vy - b) \, \nora^{-s}\\
  &= \intrm \da \intinf \db \, \intrm \dx \, f(\vx) \, \psi(\va \cdot \vx - b) \, \eta(\va \cdot \vy - b) \label{a}\\
  &= \intrm \dx \, f(\vx) \, \intrm \da \intinf \db \, \psi(\va \cdot \vx - b) \, \eta(\va \cdot \vy - b) \label{b}
\end{align}

To evaluate the integral, we introduce a change of variables. Consider the transformation from $(\va, b)$ to $(z, \vw)$:
\begin{align*}
  \begin{cases}
    z &= \va \cdot \vx - b\\
    \vw &= (a_1, a_2, \cdots a_m)
  \end{cases}
  \iff
  \begin{cases}
    b &= \va \cdot \vx - z\\
    \va &= \vw 
  \end{cases}
\end{align*}

Where ($a_1, a_2, \cdots a_m$) are the components of $\va$.

\clearpage

The Jacobian matrix of this transformation is given by:

\begin{align*}
  J &=
  \begin{pmatrix}
    \pdiffl{z}{a_1} & \pdiffl{w_1}{a_1} & \cdots & \pdiffl{w_{m-1}}{a_1} & \pdiffl{w_m}{a_1} \\
    \pdiffl{z}{a_2} & \pdiffl{w_1}{a_2} & \cdots & \pdiffl{w_{m-1}}{a_2} & \pdiffl{w_m}{a_2} \\
    \vdots         & \vdots           & \ddots & \vdots               & \vdots           \\
    \pdiffl{z}{a_m} & \pdiffl{w_1}{a_m} & \cdots & \pdiffl{w_{m-1}}{a_m} & \pdiffl{w_m}{a_m} \\
    \pdiffl{z}{b}   & \pdiffl{w_1}{b}   & \cdots & \pdiffl{w_{m-1}}{b}   & \pdiffl{w_m}{b}
  \end{pmatrix}
  \\[1.5ex]
  &=
  \begin{pmatrix}
    x_1 & 1 & 0 & \cdots & 0 & 0 \\
    x_2 & 0 & 1 & \cdots & 0 & 0 \\
    \vdots & \vdots & \vdots & \ddots & \vdots & \vdots \\
    x_{m-1} & 0 & 0 & \cdots & 1 & 0 \\
    x_m & 0 & 0 & \cdots & 0 & 1 \\
    -1 & 0 & 0 & \cdots & 0 & 0
  \end{pmatrix}
\end{align*}

Where ($x_1, x_2, \cdots x_m$) are the components of $\vx$. 
The determinant is computed by cofactor expansion along the last row. Let $C_{(i,j)}$ denote the cofactor corresponding to the element in the $i$-th row and $j$-th column. Since the only nonzero element in the last row is in position $(m+1, 1)$, its sign is given by $(-1)^{(m+1)+1}$.

The minor corresponding to this entry is the $m \times m$ identity matrix, so its cofactor is:
\begin{align*}
C_{((m+1),1)} = (-1)^{(m+1)+1} \cdot \det(I_m) = (-1)^{m+2} \cdot 1 = (-1)^{m+2}
\end{align*}

Hence:
\begin{align*}
\det(J) =-1 \cdot C_{((m+1), 1)}= -1 \cdot (-1)^{m+2} = (-1)^{m+3}, \quad \text{so} \quad |\det(J)| = 1
\end{align*}.

The integral after the transformation is as follows:

\begin{align}
  \text{(RHS)} &= \intrm \dx \, f(\vx) \, \intrm \dw \intinf \dd z \, \psi(z) \, \eta(\vw \cdot (\vy - \vx) + z)\\
  &= \intrm \dx \, f(\vx) \, \intrm \dw \intinf \dd z \, \left( \intinf \dd \z \, \psihat(\z) e^{i\z z}\right) \, \eta(\vw \cdot (\vy - \vx) + z) \label{c}\\
  &= \intrm \dx \, f(\vx) \, \intrm \dw \intinf \dd z \, \intinf \dd \z \, \psihat(\z) e^{i\z z} \, \eta(\vw \cdot (\vy - \vx) + z)\label{d}\\
  &= \intrm \dx \, f(\vx) \, \intrm \dw \intinf \dd \z \, \psihat(\z) \intinf \dd z \,e^{i\z z} \, \eta(\vw \cdot (\vy - \vx) + z)\label{e}
\end{align}

In equation \eqref{c}, we apply the Fourier inversion formula to $\psi$, which yields the integral representation of $\psi$ via $\psihat$.

\bigskip

Next, consider the following transformation:
\begin{align}
  z' = \vw \cdot (\vy - \vx) + z \\
  \iff z = z' - \vw \cdot (\vy - \vx)
\end{align}

\begin{align}
  \text{(RHS)} &= \intrm \dx \, f(\vx) \, \intrm \dw \intinf \dd \z \, \psihat(\z) \intinf \dd z' \, e^{i\z z'} \, e^{- i\z \vw \cdot (\vy - \vx)}\, \eta(z')\\
  &= \intrm \dx \, f(\vx) \, \intrm \dw \intinf \dd \z \, \psihat(\z) \, e^{- i\z \vw \cdot (\vy - \vx)} \intinf \dd z' \, \eta(z') \, e^{i\z z'} \\
  &= \intrm \dx \, f(\vx) \, \intrm \dw \intinf \dd \z \, \psihat(\z) \, e^{- i\z \vw \cdot (\vy - \vx)} \, \etahat(\z)\\
  &= \intrm \dx \, f(\vx) \, \intinf \dd \z \, \psihat(\z) \, \etahat(\z) \intrm \dw \, e^{- i\z \vw \cdot (\vy - \vx)}\\
\end{align}

Considering the transformation from $\z\vw$ to $\vw'$, and noting that its Jacobian is $\absz^m$:

\begin{align}
  \text{(RHS)} &= \intrm \dx \, f(\vx) \, \intinf \dd \z \, \psihat(\z) \, \etahat(\z) \intrm \frac{\dw}{\absz^m} \, e^{- i\vw' \cdot (\vy - \vx)}\\
  &= \intrm \dx \, f(\vx) \, \intinf \dd \z \, \dfrac{\psihat(\z) \, \etahat(\z)}{\absz^m} \intrm \dw \, e^{- i\vw' \cdot (\vy - \vx)}\\
  &= \intrm \dx \, f(\vx) \, \intinf \dd \z \, \dfrac{\psihat(\z) \, \etahat(\z)}{\absz^m} \, \delta(\vy - \vx)\\
  &=  \intinf \dd \z \, \dfrac{\psihat(\z) \, \etahat(\z)}{\absz^m} \, \intrm \dx \, f(\vx)  \, \delta(\vy - \vx)\\
  &= f(\vy) \, \intinf \dd \z \, \dfrac{\psihat(\z) \, \etahat(\z)}{\absz^m}. 
\end{align}

Thus, it is shown that the reconstruction formula is possible.

\clearpage


\section{Examples}
\subsection{Fourier Transform of the active function $\eta(z)$}

We choose the hyperbolic tangent function as the activation function $\eta$.
The Fourier transform of the hyperbolic tangent function is given by:
\begin{align}
  \etahat(\z) = \frac{-i \pi}{\sinh\left(\frac{\pi}{2} \z\right)}.
\end{align}

Now, we will show how its Fourier transform is derived.

\begin{align*}
  \etahat(\z) &= \text{p.v.}\, \intinf \tanh(z) \, e^{-i\z z} \, \dd z \\
  &= \lim_{R \to \infty} \int_{-R}^{R} \tanh(z) \, e^{-i\z z} \, \dd z 
\end{align*}

Here, p.v. denotes the Cauchy principal value. The necessity of using the Cauchy principal value arises from the fact that $\tanh(z)$ converges to a constant at infinity, whereas the exponential term exhibits oscillatory behavior.

As direct evaluation of this integral is difficult, we resort to considering a complex integral along a contour $\mathcal{C}$.

\begin{align*}
  \int_{\mathcal{C}} \tanh(z) \, e^{-i\z z} \, \dd z &= \int_{\gamma_R} \tanh(z) \, e^{-i\z z} \, \dd z + \int_{\ell_R} \tanh(z) \, e^{-i\z z} \, \dd z.
\end{align*}

Where $\mathcal{C}$ is a closed contour in the complex plane, and $\gamma_R$ and $\ell_R$ are specific segments of this contour. For $\zeta > 0$, we choose $\mathcal{C}$ to be the semi-circular contour in the upper half-plane,
consisting of the real axis segment $[-R, R]$ (denoted by $\ell_R$) and the semi-circular arc $\gamma^+_R = \{ z = R e^{i\theta} \mid 0 \leq \theta \leq \pi \}$. Conversely, for $\zeta < 0$,
we choose the semi-circular contour in the lower half-plane $\gamma^-_R = \{ z = R e^{-i\theta} \mid 0 \leq \theta \leq \pi \}$.

\bigskip

Now, we will utilize the Residue Theorem to evaluate this complex integral. 
The next steps involve:

\begin{enumerate}
  \item Identifying the contribution from the straight line segment.
  \item Calculating the residues.
  \item Showing the integral along the semi-circular arc vanishes.
\end{enumerate}

Next, we will calculate the residues at the poles within our chosen contour. The fact that poles of $\tanh(z)e^{-i\zeta z}$ are located at $z_k = \pm i (/2 + k)\pi$ for $k \in \mathbb{Z}$ can be readily seen from the following lemma:

\begin{lem}[Infinite Product Expansion of Hyperbolic Cosine]
The hyperbolic cosine function $\cosh(z)$ has the infinite product expansion:
$$ \cosh(z) = \prod_{k=0}^{\infty} \left(1 + \frac{z^2}{\left(k+\dfrac{1}{2}\right)^2\pi^2}\right) $$
\end{lem}

Since these poles are simple poles, we can use the following lemma to calculate the residues:

\begin{lem}[Residue at a Simple Pole] \label{lem:simple_pole_res}
Let $f(z) = \dfrac{p(z)}{q(z)}$ be a complex function where $p(z)$ and $q(z)$ are analytic functions. If $q(z_0) = 0$ and $q'(z_0) \neq 0$ (implying that $z_0$ is a simple zero of $q(z)$), and $p(z_0) \neq 0$, then $f(z)$ has a simple pole at $z=z_0$, and its residue is given by:
$$ \text{Res}(f(z), z_0) = \frac{p(z_0)}{q'(z_0)} $$
\end{lem}

\textbf{Case 1: $\zeta > 0$}
For $\z > 0$, we choose the semi-circular contour in the lower half-plane. Therefore, we need to consider the poles $z_k = -i (\pi/2 + k\pi)$ for $k \in \mathbb{N}$.
The residues at these poles can be computed using Lemma \ref{lem:simple_pole_res}. Here, $p(z) = \sinh(z)e^{-i \z z}$ and $q(z) = \cosh(z)$.
The residue at the pole $z_k = -i (\pi/2 + k\pi)$ is given by:
\begin{align*}
  \text{Res}(\tanh(z)e^{-i \z z}, z_k) &= \frac{\sinh(z_k)e^{-i \z z_k}}{\cosh'(z_k)} \\
  &=\frac{\sinh(z_k)e^{-i \z z_k}}{\sinh(z_k)}\\
  &= e^{-i \z z_k} \\
  &= e^{-i \z (-i (\pi/2 + k\pi))}\\
  &= e^{\z (\pi/2 + k\pi)}.
\end{align*}

As $R \to \infty$, the contour encloses an infinite number of poles in the lower half-plane, specifically $z_j = -i (j + 1/2)\pi$ for $j \in \mathbb{N}$. Thus, the total sum of residues will be an infinite series:

\begin{align*}
  \sum_{j=0}^{\infty} \Res(\tanh(z)e^{-i \z z}, z_j) &= \sum_{j=1}^{\infty} e^{\z (j + 1/2)\pi} \\
  &= e^{\z \pi/2} \sum_{j=1}^{\infty} e^{j\z \pi} \\
  &= e^{\z \pi/2} \frac{e^{\z \pi}}{1 - e^{\z \pi}}\\
  &=  \frac{i\pi}{2\sinh\left(\frac{\pi}{2} \z\right)}.
\end{align*}

It's important to note that when integrating along a contour in the lower half-plane, the contour is typically traversed anticlockwise.
Since we are integrating clockwise, a minus sign must be included when using the residue theorem

\bigskip

\textbf{Case 2: $\zeta < 0$}
By similar calculations, the infinite sum of the residues for $\z < 0$ is given by:

\begin{align*}
  \sum_{j=0}^{\infty} \Res(\tanh(z)e^{-i \z z}, z_j) &= \sum_{j=1}^{\infty} e^{-\z (j + 1/2)\pi} \\
  &= e^{-\z \pi/2} \sum_{j=1}^{\infty} e^{-j\z \pi} \\
  &= e^{-\z \pi/2} \frac{e^{-\z \pi}}{1 - e^{-j\z \pi}}\\
  &= \frac{-i\pi}{2\sinh\left(\frac{\pi}{2} \z\right)}
\end{align*}

\bigskip

Next, we will show that the integral over a semicircle is zero.

\bigskip

\textbf{Case 1: $\zeta > 0$}

For $\z > 0$, we consider the semicircular contour in the lower half-plane, denoted as $\gamma^-_R = \{ z = R e^{-i\theta} \mid 0 \leq \theta \leq \pi \}$.
We need to show that the integral over this semicircle vanishes as $R \to \infty$.

We have:
\begin{align*}
  \left|\int_{\gamma^-_R} \tanh(z)e^{-i \z z} \, \dd z\right| &= \left| \int_{0}^{\pi} \tanh(R e^{-i\theta}) e^{-i \z R e^{-i\theta}} (-i R e^{-i\theta}) \, \dd\theta \right| \\
  &\leq \int_{0}^{\pi} \left| \tanh(R e^{-i\theta}) e^{-i \z R e^{-i\theta}} (-i R e^{-i\theta}) \right| \, \dd\theta \\
  &=  R \int_{0}^{\pi} \left| \tanh(R e^{-i\theta})  \right|e^{ -\z R \sin \theta} \, \dd\theta \\
  &\leq  R \int_{0}^{\pi} e^{ -\z R \sin \theta} \, \dd\theta.
\end{align*}

Here, we use the following lemma:

\begin{lem}[Laplace's Method]
Let $f(x)$ be a twice continuously differentiable function on the interval $[a, b]$,
 such that it attains its maximum value at a unique point $x_0 \in (a, b)$, i.e., $f(x_0) = \max_{a \leq x \leq b} f(x)$, and $f''(x_0) < 0$.
 Under these conditions, we have:
\begin{align*}
  \int_a^b e^{n f(x)} \, \dd x \sim \sqrt{\frac{2\pi}{n\left|f''(x_0)\right|}}  e^{n f(x_0)} \quad \text{as } n \to \infty.
\end{align*}
Here, $\sim$ denotes that the ratio of the two sides approaches 1 as $n \to \infty$.
\end{lem}
Using Laplace's method, we can evaluate the integral as follows:
\begin{align*}
  \int_{0}^{\pi} e^{ -\z R \sin \theta} \, \dd\theta &\sim \sqrt{\frac{2\pi}{\z R \left| -\cos(0) \right|}} e^{ -\z R \sin(\frac{\pi}{2})} \\
  &= \sqrt{\frac{2\pi}{\z R}} e^{ -\z R} \\
  &= \sqrt{\frac{2\pi}{\z R}} e^{ -\z R}.
\end{align*}

Hence, we calculate the integral over the semicircle as follows:
\begin{align*}
  R \int_{0}^{\pi} e^{ -\z R \sin \theta} \, \dd\theta &\leq R \sqrt{\frac{2\pi}{\z R}} e^{ -\z R} \\
  &= \sqrt{\frac{2\pi R}{\z}} e^{ -\z R}.
\end{align*}

Therefore, we conclude that:
\begin{align*}
  \int_{\gamma^+_R} \tanh(z)e^{-i \z z} \, \dd z &\to 0 \quad \text{as } R \to \infty.
\end{align*}

\bigskip

\textbf{Case 2: $\zeta < 0$}
For $\z < 0$, we consider the semicircular contour in the upper half-plane, denoted as $\gamma^+_R = \{ z = R e^{i\theta} \mid 0 \leq \theta \leq \pi \}$.
We need to show that the integral over this semicircle vanishes as $R \to \infty$.
We have:
\begin{align*}
  \left|\int_{\gamma^+_R} \tanh(z)e^{-i \z z} \, \dd z\right| &= \left| \int_{0}^{\pi} \tanh(R e^{i\theta}) e^{-i \z R e^{i\theta}} (i R e^{i\theta}) \, \dd\theta \right| \\
  &\leq \int_{0}^{\pi} \left| \tanh(R e^{i\theta}) e^{i \z R e^{i\theta}} (i R e^{i\theta}) \right| \, \dd\theta \\
  &\leq \int_{0}^{\pi} \left| \tanh(R e^{i\theta}) e^{i \z R e^{i\theta}} (i R e^{i\theta}) \right| \, \dd\theta \\
  &=  R \int_{0}^{\pi} \left| \tanh(R e^{i\theta})  \right|e^{ i \z R \sin \theta} \, \dd\theta \\
  &\leq  R \int_{0}^{\pi} e^{\z R \sin \theta} \, \dd\theta.
\end{align*}

Here, using Laplace's lemma, the same calculation yields 0.
Finally, we can conclude that the integral over the semicircle vanishes as $R \to \infty$.

\bigskip

Thus, we have:
\begin{align*}
  \etahat(\z) &= \frac{-i\pi}{2\sinh\left(\frac{\pi}{2} \z\right)}.
\end{align*}

\subsection{Inverse Fourier Transform of the dual function $\psi(z)$}

We choose the following function as the dual function$\psi$ in the Fourier domain:

\begin{align}
  \psihat(\z) = (i \z)^{2k-1} e^{-\z^2}, \quad k \in \mathbb{N}.
\end{align}

Now, we calculate its inverse Fourier transform of $\psihat(\z)$:

\begin{align*}
  \psi(x) &= \int_{-\infty}^{\infty} \psihat(\z) e^{i \z x} \, \dd\z \\
  &= \int_{-\infty}^{\infty} (i\z)^{2k-1} e^{-\z^2} e^{i \z x} \, \dd\z.
\end{align*}

To evaluate this integral, let us consider the following integral:
\begin{align}
  I(z) = \int_{-\infty}^{\infty} e^{-\z^2} e^{i \z z} \, \dd\z.
\end{align}
We can evaluate this integral by completing the square in the exponent:
  \begin{align*}
    I(z) &= \int_{-\infty}^{\infty} e^{-\z^2} e^{i \z z} \, \dd\z 
    = e^{-z^2/4} \int_{-\infty}^{\infty} e^{-\frac{1}{2}(\z - iz)^2} \, \dd\z
  \end{align*}

  By a change of variable, $u = \z - iz$, and using the well-known Gaussian integral formula, we obtain:

  \begin{align*}
    I(z) &= \sqrt{2\pi} \, e^{-z^2/4}.
  \end{align*}

Next, we differentiate the expression for $I(z)$ with respect to x and apply the Leibniz integral rule (differentiation under the integral sign). Differentiating $(2k-1)$ times, we have:
  \begin{align*}
    \diff{^{2k-1}}{z^{2k-1}} I(z) &= \diff{^{2k-1}}{z^{2k-1}} \int_{-\infty}^{\infty} e^{-\z^2} e^{i \z z} \, \dd\z \\
    &= \int_{-\infty}^{\infty} e^{-\z^2} \diff{^{2k-1}}{z^{2k-1}} e^{i \z z} \, \dd\z \\
    &= \int_{-\infty}^{\infty} e^{-\z^2} (i \z)^{2k-1} e^{i \z z} \, \dd\z \\
  \end{align*}

  On the other hand, differentiating the result of the integral $(2k-1)$ times gives:
  \begin{align*}
    \diff{^{2k-1}}{z^{2k-1}} I(z) &= \diff{^{2k-1}}{z^{2k-1}} \sqrt{2\pi} \, e^{-z^2/4} \\
    &= \sqrt{2\pi} \, \diff{^{2k-1}}{z^{2k-1}} e^{-z^2/4} \\
    &= \sqrt{2\pi} \, (-1)^{k-1} \left(\frac{1}{2}\right)^{2k-1} H_{2k-1}\left(\frac{z}{2}\right) e^{-z^2/4}. 
  \end{align*}
  Where, $H_n(x)$ is the Hermite polynomial defined by:
  \begin{align}
    H_n(x) = (-1)^n e^{x^2} \diff{^n}{x^n} e^{-x^2}.
  \end{align}

  The $(2k-1)$-th derivative of $e^{-x^2/4}$ is proportional to the Hermite polynomial $H_{2k-1}(x/2)$, which is a known result. Combining these two expressions, we can relate the inverse Fourier transform to the Hermite polynomial.

\clearpage

\nocite{*} 
\bibliographystyle{plain} 
\bibliography{ridgelet_reconstruction_note_v1} 

\end{document}
\nocite{*} 
\bibliographystyle{plain} 
\bibliography{ridgelet_reconstruction_note_v1} 

\end{document}